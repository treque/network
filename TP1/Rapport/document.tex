\documentclass{article}
\usepackage{siunitx}
\usepackage[utf8]{inputenc}
\usepackage{amsmath}
\usepackage[demo]{graphicx}
\usepackage{geometry}

\usepackage{listings}
\usepackage{color}

\definecolor{dkgreen}{rgb}{0,0.6,0}
\definecolor{gray}{rgb}{0.5,0.5,0.5}
\definecolor{mauve}{rgb}{0.58,0,0.82}

\lstset{frame=tb,
	language=Java,
	aboveskip=3mm,
	belowskip=3mm,
	showstringspaces=false,
	columns=flexible,
	basicstyle={\small\ttfamily},
	numbers=none,
	numberstyle=\tiny\color{gray},
	keywordstyle=\color{blue},
	commentstyle=\color{dkgreen},
	stringstyle=\color{mauve},
	breaklines=true,
	breakatwhitespace=true,
	tabsize=3
}

\graphicspath{{Images/}}
\title{Travail pratique 1}
\author{Huyen Trang Dinh, Myriam Kiriakos}
\date{21 octobre 2019}

\begin{document}

 
 \begin{titlepage}
 	\newgeometry{left=20mm,right=20mm,top=20mm,bottom=20mm}
 	
 	\newcommand{\HRule}{\rule{\linewidth}{0.5mm}} % Defines a new command for the horizontal lines, change thickness here
 	
 	\center % Center everything on the page
 	
 	%----------------------------------------------------------------------------------------
 	%	HEADING SECTIONS
 	%----------------------------------------------------------------------------------------
 	
 	\textsc{\Huge École Polytechnique de Montréal}\\[0.5cm] % Name of your university/college
 	\textsc{\huge Baccalauréat en génie logiciel}\\[2.3cm] % Major heading such as course name
 	\textsc{\Large }\\[0.2cm] % Minor heading such as course title
 	\textsc{}\\[0.2cm]
 	\vspace{-1cm}
 	%----------------------------------------------------------------------------------------
 	%	TITLE SECTION
 	%----------------------------------------------------------------------------------------
 	\text{}\\
 	\HRule \\[0.6cm]
 	{ \huge \bfseries Travail Pratique 1}\\[0.4cm]
 	{ \LARGE INF3405 - Réseaux Informatiques}\\[0.4cm] % Title of your document
 	\HRule \\[1.5cm]
 	\vspace{3cm}
 	
 	
 	présenté à\\
 	Bilal Itani
 	
 	\vspace{4cm}
 	%----------------------------------------------------------------------------------------
 	%	AUTHOR SECTION
 	%----------------------------------------------------------------------------------------
 	
 	{\large
 		\emph{Par:}
 		\vspace{0.2cm}\\ % Your name
 		\textsc{Myriam Kiriakos ()\\}
 		\textsc{Huyen Trang Dinh (1846776)\\} % Your name
 		
 	}
 	
 	\vspace{3cm}
 	
 	% If you don't want a supervisor, uncomment the two lines below and remove the section above
 	%\Large \emph{Author:}\\
 	%John \textsc{Smith}\\[2cm] % Your name
 	
 	%----------------------------------------------------------------------------------------
 	%	DATE SECTION
 	%----------------------------------------------------------------------------------------
 	
 	\textsc{{\large 21 octobre 2019}}\\[2cm] % Date, change the \today to a set date if you want to be precise
 	\vspace{-1.2cm}

 	\vfill % Fill the rest of the page with whitespace
 	
 \end{titlepage}
 
 \restoregeometry
 \pagebreak
 
 
\section{Introduction}
Mettre en contexte les objectifs du tp
\begin{itemize}
  \item la position du référentiel du drone par rapport à celui du laboratoire $pos$ en \si{\metre}
  \item l'angle de rotation $mu$ du système $\mu$ en \si{\radian} autour de l'axe $OY$
  \item le vecteur $va$ représentant la vitesse angulaire $\vec{\omega}$ du système autour de son centre de masse en \si{\radian\per\second}
  \item un vecteur de $f_i$ représentant le coefficient du moteur $M_i$
\end{itemize}


\section{Présentation}
de vos travaux. Une explication de votre solution mettant en lumière la prise en compte des
principaux requis du système. Si vous utilisez des configurations particulières des bibliothèques ou des
projets, précisez-les également
\subsection{Centre de masse}

Il faut tout simplement multiplier cette matrice à gauche du vecteur à tourner.
\subsection{Moment d'inertie}
ablalablbal
\begin{lstlisting}
// Hello.java
import javax.swing.JApplet;
import java.awt.Graphics;

public class Hello extends JApplet {
		public void paintComponent(Graphics g) {
			g.drawString("Hello, world!", 65, 95);
		}    
}
\end{lstlisting}
\section{Difficultés rencontrées}
\subsection{Simulation 1}
talk solutions too

 \section{Critiques et améliorations}


\section{Conclusion}
: Expliquez en quoi ce laboratoire vous a été utile, ce que vous avez appris, si vos attentes ont
été comblées, etc.


\end{document}
